\tikzset{
  host/.style={
    scale=\nodescale,
    inner sep=1pt,
    draw,
    rounded rectangle,
  },
  edge/.style={
    scale=\nodescale,
    inner sep=1pt,
    draw,
    rounded rectangle,
    fill=blue!20,
  },
  core/.style={
    scale=\nodescale,
    inner sep=1pt,
    draw,
    rounded rectangle,
    fill=green!30,
  },
  port/.style = {
    % font=\tiny,
    scale=.5,
    inner sep=1pt,
    draw,
    signal,
    fill=white,
  }
}
\newcounter{previtem}

\section{Implementing and Testing a Probing-Based Path Verification (PBV)} \label{sec:implementation}

PBV consists of sending special packets (probes) with path-verification capabilities. The path verification system used is described in \autoref{sec:multisigmodel}. Some assumptions are made:
\begin{inlinelist}
    \item Each node is assumed to be secure, that is, no node will alter the packet in any way that is not expected. This is a common assumption in SDN networks, where a trusted party is the only entity that can alter the network state;
    \item  Every link is assumed to be perfect, that is, no packet loss, no packet corruption, and no packet duplication;
    \item  Protocol boundary is IPv4, this means that 
    %BETA
    %\polka
    \pathsec is only used inside this network, and only IPv4 is used outside; \label{assumption:ipv4}
    \item All paths are assumed to be valid and all information correct unless stated otherwise.
\end{inlinelist}
Furthermore, this implementation is a proof-of-concept, and there is little concerns about performance other protocol support. The tests were made using an ICMP (ping) packet. The following subsections detail the setup and implementation.

\subsection{Setup}

\pIV is a language used to program the data plane of network devices \cite{p4}. Behavioral Model 2 (\bmv) is a software switch that supports the \pIV language \cite{bmv2}. Mininet is a network emulation tool that allows the creation of virtual networks with a controlled environment \cite{mininet}. Mininet-wifi\cite{mininet-wifi} is a Mininet fork used in this work due to its compatibility with \bmv. Despite the name, no Wi-Fi emulating feature was used.

The Controller is responsible for the network setup and for calculating reference signatures.
In our implementation, the Python Script that controls Mininet acts a controller, and itself is not a network entity in Mininet.
Scapy is a Python library that allows the creation and manipulation of packets \cite{scapy}. It is used to sniff packets both for Controller operation and in the experiments and automate tests. 
The Tester is responsible for triggering Mininet hosts to send packets, setting up Scapy for parsing them and then asserting their signatures to Controller-calculated references. Only Scapy's sniffing capabilities were used, and tests were performed with a regular \texttt{ICMP} packet (\textit{ping}).
The setup is shown in \autoref{setup}. 


\subsection{Implementation}

We define the header fields \timestamp and \lhash, both 32-bits wide, to indicate the uniquely-generated number and the current lightweight multi-signature, respectively.

To maintain \polka protocol compatibility, probe packets are identified by a different version number in \polka's \texttt{version} field, so interoperates with \polka depending on the \texttt{version} header. Regular \polka packets use version \texttt{0x01}, and probe packets (with \timestamp and \lhash fields) use version \texttt{0xF1}.

\autoref{topology:base} shows the used topology used in the experiments where $s_i$ is a switch, $e_i$ is an edge node, and $h_i$ is a host. The topology is a linear chain with 10 switches and 11 hosts. The links are bidirectional, but only one direction is shown for simplicity. %The ports are numbered from $1$ to $3$, and the ports used are shown in the figure. The topology is used to test the path verification mechanism.

\begin{figure}[ht]
    \centering
    \begin{subfigure}[b]{0.49\linewidth}
    \centering
    \begin{tikzpicture}[
        outer sep=auto,
        every node/.style={draw, rectangle, label distance=-4pt},
        switch/.style={chamfered rectangle, fill=green!20, inner sep=1pt},
        % every fit/.style={rectangle,draw}
    ]
        \node (controller) {Controller};
        \node[right=12pt of controller] (tester) {Tester};
        \draw[->] (controller) to (tester);
        \node[label=above:Python script, dotted,
          fit={(controller) (tester)}] (py) {};

        \node[switch, below=32pt] at (py) (s1) {cores};
        \node[switch, left=12pt] at (s1.west) (ingress) {ingress};
        \node[switch, right=12pt] at (s1.east)      (egress) {egress};
        \draw[->] (ingress) to (s1);
        \draw[->] (s1) to (egress);
        \node[label=below:Mininet, dotted,
            fit={(ingress) (s1) (egress)}] {};

        \draw[->] (ingress) -- node[fill=white, inner sep=1pt] {\footnotesize Scapy} (controller);
        \draw[->] (egress) -- node[fill=white, inner sep=1pt] {\footnotesize Scapy} (tester) ;
    \end{tikzpicture}
    \caption{Experimentation setup.\\Arrows show data flow.}
    \label{setup}
    \end{subfigure}
    \hfill
    \begin{subfigure}[b]{0.49\linewidth}
    \centering    
    \newcommand{\nodescale}{0.8}
    \newcommand{\nswitches}{10}
    \newcommand{\newtopoitem}[2]{%
        \node[host, right=2pt] (H#1) at (H#2.east)  {$h_{#1}$};
        \node[edge, above=4pt] (E#1) at (H#1.north) {$e_{#1}$};
        \node[core, above=4pt] (S#1) at (E#1.north) {$s_{#1}$};
        \draw (H#1) -- (E#1) -- (S#1);
    }
    \setcounter{previtem}{11}
    \begin{tikzpicture}[outer sep=auto]
        \node[host] (H11) {$h_{11}$};
        \foreach \x in {1, ..., \nswitches} {
            \newtopoitem{\x}{\arabic{previtem}}
            \ifnum \theprevitem < 11
                \draw (S\x) to (S\theprevitem);
            \fi
            \setcounter{previtem}{\x}
        }
        \draw (H11) to (E1);
            
    \end{tikzpicture}
    \caption{Baseline topology used in experimentation.}
    \label{topology:base}
    \end{subfigure}
    \hfill
      \vspace{0.5cm}
% \end{figure}
% Joins both figures into one
% \begin{figure}[ht]
%     \centering
    \tikzset{
        outer sep=auto,
        every node/.style={draw, chamfered rectangle, label distance=-4pt, inner sep=2pt},
    }
    \begin{subfigure}[b]{0.49\linewidth}
    \centering    
    \begin{tikzpicture}
        \node (parse) {Parse};
            
        \node[above right=24pt] (encap) at (parse.east) {Encapsulation};
        \node[right=12pt] (core) at (encap.east) {SDN core};
        \draw[->, very thick] (parse.east) -- (encap)
            node[draw=none,pos=0.7,xshift=2pt,anchor=mid west] {\footnotesize if \texttt{0x0800} (ethernet, do ingress)};
        \draw[->, very thick] (encap) -- (core);
        
        \node[below right=24pt] (decap) at (parse.east) {Decapsulation};
        \node[right=12pt] (out) at (decap.east) {external net};
        \draw[->, very thick] (parse.east) -- (decap)
            node[draw=none,pos=0.7,xshift=2pt, anchor=mid west] {\footnotesize if \texttt{0x1234} (polka, do egress)};
        \draw[->, very thick] (decap) -- (out);
    \end{tikzpicture}
    \caption{Edge ingress/egress pipeline.}
    \label{fig:pipeline-edge}
    \end{subfigure}
    \hfill
    \begin{subfigure}[b]{0.49\linewidth}
    \centering    
    \begin{tikzpicture}
        \node (parse) {Parse};
        \node[right=12pt] (nhop) at (parse.east) {nHop};
        \node[right=12pt] (sign) at (nhop.east) {Sign};
        \node[right=12pt] (deparse) at (sign.east) {Deparse};

        \draw[->, very thick] (parse) -- (nhop);
        \draw[->, very thick] (nhop) -- (sign);
        \draw[->, very thick] 
            (nhop.east) ..controls ++(0,-1) and ++(0,-0.5) .. (deparse.195) node[midway,below=-4pt,draw=none] {\footnotesize if not probe};
        \draw[->, very thick] (sign) -- (deparse);
            
    \end{tikzpicture}
    \caption{Data plane core pipeline.}
    \label{fig:pipeline-core}
    \end{subfigure}
    \caption{Implementation description.}
    \label{fig:implementation}
\end{figure}

Edge nodes can receive both \polka packets or IPv4 packets. If the IPv4 protocol \texttt{ethertype} header field is detected (\texttt{0x0800}), it must be a packet from outside the network, so it proceeds with the role of an ingress edge and \polka headers need to be added. Let this process be called \textit{encapsulation}. If a \polka protocol \texttt{ethertype} field is detected (\texttt{0x1234}), it must be a packet from the network core, so it proceeds with the role of an egress edge and the original IPv4 packet must be unwrapped. Let this process be called \textit{decapsulation}. This flow is illustrated in \autoref{fig:pipeline-edge}.

\polka headers consists of the route polynomial (\routeid), along with \texttt{version}, \texttt{ttl} and \texttt{proto} (stores the original \texttt{ethertype}).
% \routeid and hop calculation is out of the scope of this paper.
During encapsulation, the packet can be made into a probe packet by setting the \texttt{version} header field to \texttt{0xF1} and further defining the 32-bit fields \timestamp to any unique value and \lhash initially equal to \timestamp.

On the core network, illustrated on \autoref{fig:pipeline-core}, the hash function used in the \textit{Sign} step is \textit{SipHash-2-4-32}\cite{siphash} (\siphash). It is expected to provide maximum MAC security possible for $c \geq 2$ and $d \geq 4$. The function is not made to be collision-resistant, but its features are well-suited to be used as a MAC\cite{siphash}. 
Hash collisions for a small scale test like we present is extremely unlikely, since our hash output is evaluated to be indistinguishable from a uniform random function, and collisions would need to happen in the same route for impacts to occur.
%BETA
%which 
After calculating the next output port (\nhop) and if the packet is a probe packet, \lhash is computed as
$$
\lhash \leftarrow \textit{SipHash2-4-32}(\nodeid || \nhop || \timestamp || \texttt{0000000}, \lhash)
$$
Meaning that a 64-bit string made of concatenating \nodeid, \nhop, \timestamp with the remaining bits set to \texttt{0} for padding is used as key for the \siphash function hashing \lhash. 
