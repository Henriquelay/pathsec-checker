\section{Problem definition} \label{sec:definition} 

% Top to bottom, indo do sistema mais amplo ao programa em baixo nível


The \pathsec approach for proof-of-transit uses a logically centralized SDN Controller that selects routing paths and configures the core and edge nodes. While ingress and egress edges encapsulates/decapsulates metadata into/from packets, core nodes execute hashing operations alongside basic \polka packet forwarding.

%Calculating each hop is done exploiting the Chinese Remainder Theorem\cite{polka}, and is out of the scope of this paper. 
\pathsec is based on \polka's source routing which explores the Chinese Remainder Theorem\cite{polka}.
\polka assigns a unique route identifier (\routeid) to each path in the core network, and for each core switch it defines a unique \nodeid. Once a packet is associated to a specific route, each packet in this packet  will carry in its header the respective \routeid (embeded by the ingress edge). Then the core switches use the \routeid field and their respective \nodeids to calculate the output \portid (it is a result of the Residue Number System (RNS)-based encoding\cite{pathsec}, which is out of the scope of this paper), routing the packets to the next core node in the expected path for the \routeid.

Furthermore, \pathsec designs a simplified signing mechanism using the uniqueness of the switch-port pair (\nodeid, \portid) for each \routeid, and by holding these components in as a secret, they can be used together as keys for cryptographic hashing functions. Every switch-port pair acts as a native secret information by not being shared in the resulting message, enabling a lightweight multi-signature scheme based on Hash-based Message Authentication Code (HMAC): before routing a packet, a core node replaces the lightweight signature from the previous core node (embedded in the packet header), with a new one, hashing the previous one with the pair (\nodeid, \portid). This mechanism ensures an inherent property of the routing system that facilitates efficient path verification. 

%BETA: JOGUEI PARA A SUBSEC ANTERIOR (INTRODUÇÃO)
%In the complete solution proposed by \pathsec, once packets reach the egress edges, the final lightweight multi-signatures are extracted from the packets and sent to smart contracts deployed in a blockchain. These signatures are then compared to reference signatures previously registered by a trusted party (the network controller), allowing to verify packets' \textit{proof-of-transit} and to register path compliance. Blockchain, smart contracts and signature checking in deployment conditions is out of scope for this work. For this work, the scope is to generate the signatures for packets and verify how the system behaves.


% \anotacao{Acho que esse título (ou conteúdo) pode melhorar. O \pathsec não diz explícitamente que usar o \polka, então parece "irrelevante" pro propósito do artigo}


\subsection{Problem formalization} \label{sec:formalization}
\newcommand{\Pie}{P_{i \rightarrow e}}

Let $i$ be the source node (\textbf{i}ngress node) and $e$ be the destination node (\textbf{e}gress node). Let the path from $i$ to $e$ $\Pie$ be a sequence of nodes: the path is defined by $ \Pie = (i, s_1, s_2, ..., s_{n - 1}, s_n, e) $\label{eq:path-def}, $s_n$ being the $n$-th core node in the path. In SR protocols, packets' routes through the core network are defined in $i$. That is, $i$ sets the packet header with enough information for each core node to calculate the next hop. 


The main problem we are trying to solve is path verification, that is, to have a way to ensure if the packet follows the path defined. A solution must be able to identify if the packet:
\begin{inlinelist}
    \item has passed through all nodes in the Path;
    \item has passed through the correct order of nodes;
    \item has \textbf{not} passed through any node that is \textbf{not} in the path.
\end{inlinelist}




More formally, given a routed sequence of nodes $P_{i \rightarrow e}$, and a captured sequence of nodes actually traversed $P_j$, a solution must identify if $P_{i \rightarrow e} = P_j$.
Notably, it does not require validation, that is, it needs only to respond if $P_{i \rightarrow e} = P_j$ and does not need to know any of their contents or check their validity as a route.


\subsection{Multi-signature model for Path Verification} \label{sec:multisigmodel}
%cite pathsec

To balance trace effectiveness and scalability, \pathsec proposes using packets themselves as probes\cite{pathsec}\cite{PINT2020}, with a lightweight multi-signature, keeping the header size fixed. That is, each core node hashes the previous node's signature with a key only itself (and the Controller) can possess, and pass the hash result forward, replacing the previous hash results, and all following core nodes will do the same until the edge node is reached. There is a cost to this write operation, especially when analyzing relative to \polka since \polka does not require any part of the packet to be written on in the whole route, but since header size is fixed, and not all packets are expected to contain probing metadata, the cost is mitigated.

Each node's execution plan is stateless. So, in terms of signature composition, a node $n_i$ can be viewed as a function $f_{n_i} (x)$. A node can alter the header of the packet, which
%BETA
% we will use
is used to push information downstream,
%and 
allowing ultimately to detect if the path taken is correct.
%BETA
%\mycomment{Está preciso e claro dizer que ``which we will use to push information downstream''?}

In order to represent all nodes by the same function, we assign a distinct key value $k_{n}$ for each node $n$, and use a bivariate function $f(k_{n}, x) = f_{n} (x)$.
By using functions in two variables and enforcing one of the variables assume a value that's unique between nodes, we ensure that the function's result is unique for each switch as long as the function is collision resistant enough, that is, $f_{a} (x) \neq f_{b} (x) \iff a \neq b$.

Using function composition is appropriate, as it preserves the order-sensitive property of the path, since $f \circ g \neq g \circ f$ in a general case.
In our model, each node will execute a single function of this composition, using the previous node's output as input.
In this way, $ (f_{n_1} \circ f_{n_2} \circ f_{n_3})(x) = f(k_{n_3}, f(k_{n_2}, f(k_{n_1}, x))) $, $f$ being a function sufficiently resistant to collision. A cryptographic hash function is essential for maintaining the security of sensitive data. A strong avalanche effect is particularly desirable, ensuring that even minor variations in the input produce significantly different hash values. While the computational cost of cryptographic hash functions is higher compared to non-cryptographic alternatives, modern algorithms such as BLAKE2\cite{BLAKE2} and SipHash\cite{siphash} have significantly reduced this performance gap, making them viable choices for both security and efficiency.
