% Probing
\section{Introduction}
% O que foi feito
% Motivaçao (!!)
% Oq é SR e Probe-Based Verification (PBV)

% - motivação source routing
% - com isso, há a proposta do polka (sistema de resíduos sem explicar o que é precisamente)
% - mas que que há uma limitação: como comprovar que de fato a rota foi respeitada
% - foi criada a solução do pathsec
% ...
% - contexto deste trabalho: a implementação e testes do pathsec no mininet

Ever since \textit{Source Routing} (SR) was proposed, there has been a need to ensure that packets traverse the network along the paths selected by the source, not only for security reasons, but also to ensure that the network is functioning 
%BETA
%correctly
properly
and correctly configured. This is particularly important in the context of \textit{Software-Defined Networks} (SDNs), where the control plane can select paths based on a variety of criteria\cite{SRSDN}.

In this paper, we propose an implementation on Programming Protocol Independent Processors (\pIV) for 
%BETA
%Probing-Based Path Verification (PBV),
a Probing-Based Path Verification (PBV) mechanism originally proposed in \pathsec\cite{pathsec}. This mechanism provides proofs of packet forwarding, confirming that an expected route was indeed used for a packet.
This is achieved by using a composition of hash functions on stateless core switches, 
%BETA
combined with the switches native secret code (\nodeids) and the respective route's output port, which results into a light-weight multi-signature. This signature  can then be compared
%BETA
to a reference signature generated by a trusted party, the network controller. This work tests a proof-of-concept implementation under different adversity scenarios and results show the approach is effective as a proof-of-transit solution.

The structure of this paper is as follows: \autoref{sec:definition} outlines the problem definition and the proposed solution. \autoref{sec:implementation} describes the implementation details, including the architecture and the tools used. \autoref{sec:testing} presents the scenarios used to evaluate the system and the results, and \autoref{sec:conclusion} concludes the paper.

All the code and test results are available online\footnote{\url{github.com/Henriquelay/polka-halfsiphash}}.
